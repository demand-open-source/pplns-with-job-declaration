\documentclass[11pt]{article} 

\usepackage[utf8]{inputenc} 
\usepackage{amsmath}
\usepackage{hyperref}

\usepackage{geometry}
\geometry{a4paper}

\usepackage{subfig}

\title{PPLNS with Job Declaration}
\author{Lorenzo Bonazzi, Filippo Merli}
\date{August 28, 2024} % Activate to display a given date or no date (if empty),
         % otherwise the current date is printed 
         
\newtheorem{remark}{Remark}[subsection]

\begin{document}
\maketitle

\section{Introduction}

Current payout schemes for mining pools account only for provided hashrate, which is not ideal for Job Declaration.

One of the biggest issues for the Bitcoin network is the centralization of transaction selection, this issue is solved by Stratum v2 \cite{sv2}. Stratum v2 enables a miner to select their own block template of transactions (this feature is known as \emph{Job Declaration}). Since transactions fees will become the largest percentage of the block reward, the network needs a system that calculates payouts based both on selected transaction fees and hashrate provided. We propose a new payment schema, called PPLNS-JD (or informally Job Slicing), that expands PPLNS (see \cite{rosenfeld} and \cite{ocean}), so a miner can be paid fairly given hashrate provided, and fees in the mined block template. The core idea is that a miner performing Job Declaration, who are more capable of optimizing transaction selection for fee over vbytes (and consequentially in fees of the block template), should be rewarded more. PPLNS-JD can be implemented as an extension of Sv2 \cite{extension} and it is easily accountable, so that miners' trust in pool operator is minimized.

We define a fair accounting schema that splits pool payout in two parts. The first comes from the block subsidy and pays accordingly to the difficulty score of the share, in the same fashion as traditional PPLNS. The second comes from the block fees and pays accordingly to share difficulty score and the block fees in such a way that the share with the highest fee in its group of shares (slice, see below) is paid at least $F_S/n$, where $F_S$ is the part of the block fee reserved to that group of shares $S$ and $n$ is the number of shares in $S$.

In the following, a \emph{window} is a group of PPLNS shares that are paid once a block is found by the pool. So, the windows size is $N$, where $N$ is symbol used for the acronym "Pay Per Last $N$ Shares". A \emph{mining round} is the number of shares produced in the time between a Bitcoin block and the following one or, equivalently, all the shares \emph{with the same prev-hash}.

\section{Terminology}

Readers unfamiliar with bitcoin mining at a technical level may benefit from the following common terminology definitions:

\begin{enumerate}
	\item difficulty - an amount of leading zeroes in a given hash, the Bitcoin consensus rules use this when validating blocks, there is a Bitcoin wide network difficulty that is known and calculated independently by all nodes
	\item share - a hash of a certain difficulty determined by the pool but insufficient to satisfy the networks difficulty, shares are submitted by miners as proof of work to a pool, pools often dynamically set the share difficulty for miners
	\item job - data given to a miner to hash on, including but not limited to a block template
	\item work - common term when discussing difficulty of shares, work is often the difficulty of an individual share or a the sum of multiple shares difficulty
	\item block fees - a portion of every transaction paid to miners
	\item block subsidy - the predetermined supply issuance of bitcoin in a block, this value halves every 210,000 blocks, the initial subsidy was 50 bitcoin per block
	\item block reward - block subsidy + block fees
	\item hashrate - the amount of hashes per second, generally calculated using shares within a time frame, certain machines have "sticker hashrate" which is a general number to indicate the machines level of efficiency for hashing. Commonly represented as Terrahash or Gigahasg per second (TH/s or GH/s)
\end{enumerate}

\section{Slices}
In order to pay shares fairly with Job Declaration, shares must be scored by fees earned in a block template and the payout of the block fees must be distributed accordingly. Doing so, it is necessary to divide them into groups and compare scores only for shares belonging to the same group. Comparing two shares taken randomly in a traditional PPLNS window is inaccurate, because the mempool maximum extractable fees (MMEF) may change consistently. The same is true for randomly selected shares within a mining round.

We call a group of shares whose fees can be scored (and the scores can be compared) a \emph{slice}, which is a portion of the stream of shares. A slice must not cross two mining rounds and all shares in one slice must have the same prevhash. A more precise definition of a \emph{slice}

Note that within a mining round, MMEF can not decrease. The only way to remove high fees transactions from mempool is to have them mined, therefore, MMEF grows monotonically within a single mining round. If a slice consists of \emph{all} the shares with the same prevhash (the slice consists of the shares produced between two consecutive blocks), there are some incentives for miners to performs a variation of pool hopping (see Appendix).

To remove this incentive, slices must be shorter than the entire mining round. Intuitively, they must be short enough to be possible for the fees to be approximated as constant. Since a slice ends when a new slice begins, it is enough to define when a slice begins. The slices are created by the pool and the score is calculated based on fees. Each slice has a \emph{reference job}. We assign $f$ the fees of a particular reference job. Let $\delta>0$ be chosen by the pool. Then, every share that belongs to the slice for a reference job must have fees lower than \[ f+\delta.\]

A new slice is created and the reference job for the slice is updated when the pool receives a share in which:
\begin{enumerate}
	\item fees are greater than or equal to $f+\delta$ (increased MMEF)
	\item have a prevhash different from the actual slice reference job (new block)
\end{enumerate}
Every share received by the pool that does not meet condition 1 or 2 is put in the existing slice. 
%I'm not certain I articulated this correctly.%

If there are $t$ slices in the window and $F$ are the collected fees of the mined block, assign to each slice $S$ an amount $F_S = F/t$. Then, all the shares will be scored basing on the fees of selected transactions and $F_S$ will be redistributed to all the shares of $S$ accordingly with its fee score.
% I wonder if this scoring paragraph should be moved below. %

\section{Scores}
In normal PPLNS, once the pool finds a block, every share in the window is scored and the full block reward (coinbase and block fees) is distributed accordingly. This scoring assumes all miners are mining on the same job provided by the pool. With the addition of Job Declaration, this type of scoring becomes unfair. 

Given two miners with the same hashrate that submit the same amount of work to the pool, but each has different block templates via Job Declaration. The first miner optimizes transaction selection for maximum fees, and the second not so much. With traditional PPLNS, these two miners are paid equally since only hashrate (work submitted via shares) is considered. This is unfair as the miner with optimized transaction fees is working toward higher block reward for the pool, and should be rewarded commensurately.

To solve this inequity, we split the payout of each share, and score the block subsidy and the block fee differently. Since the block reward is determined independent of other transactions in the block, we also consider it independently from Job Declaration. We distribute the block reward based on traditional PPLNS share scoring: $score_d(s_i)$ of the $i$-th share $s_i$.\newline
For Job Declaration, we must introduce a new method to calculate $score_d(s_i)$ based on the fees for a submitted share, while retaining the score based on the work of submitted shares. This is desireable as the block subsidy trends to zero and fees become the predominate value of a block reward. 

In the final subsection precise definition for the payout of $s_i$ is given.

\subsection{Difficulty-based score}
If the window contains $N$ shares, then for the $i$-th share $s_i$ we define its score
\[score_d(s_i) = \frac{d_i}{\sum_{j=1}^Nd_j},\]
where $d_j$ is the difficulty of the share $s_j$ (for example, Ocean TIDES\cite{ocean} suggest $N$ be the sum at the denominator eight times the bitcoin difficulty).\newline
\begin{remark}
	Note that \[\sum_{j=1}^N score_d(s_i) = 1\] depends only on difficulty.
\end{remark}

\subsection{Fee-based score}
Suppose we want to score the $i$-th share $s_i$, and suppose that this share belongs to a slice $S$ that contains all the shares from the $k_1$-th to the $k_2$-th.
For the $i$-th share, $\bar d_i = score_d(s_i)$ is the difficulty-based score (calculated as above) and $f_i$ the fees of the share (which depends on the transactions chosen in the template). We define the score relative to the fee
\[
	score_f(s_i) = \frac{\bar d_if_i}{\sum_{j=k_1}^{k_2}\bar d_j f_j}.
\]
\begin{remark}\label{properties-diff-based-score}
	\begin{enumerate}
		\item The dependence on difficulty-based scoring is intentional so two shares with the same fee also account for computational work needed.
		\item A slice $S$ contains $n$ shares, all of the same difficulty score. If $f_i = f_{max}$ is the max fee of every share in the slice $S$, then the payout for this share is at least $score_f(s_i) \ge 1/n$. This sentence is proven by contradiction. So, assume that
		% I don't understand what "contradiction" means here
		      \[score_f(s_i) < 1/n.\]
		      Since all the shares have the same difficulty score, we have
		      \[\frac{\bar d_if_{max}}{\sum_j d_j f_j} = \frac{f_{max}}{\sum_j f_j}.\]It follows that
		      \[ \frac{f_{max}}{\sum_j f_j} <\frac{1} {n}.\]
		      So,
		      \[ 1 = \frac{\sum_k f_k}{\sum_j f_j} = \sum_k \frac {f_k}{\sum_j f_j} \le \sum_k \frac{f_{max}}{\sum_j f_j}<\sum_k\frac{1}{n} = 1\]
		      which is impossible.
		\item if there is $c>0$ such that $f_i = cf_j$ for some $s_i, s_j$ two shares in the slice $S$, then $payout_f(s_i) = c \cdot payout_f(s_j)$. This means that if a miner becomes $c$ times more capable of finding a template with higher fees, the shares will be paid proportionally.
		\item similarly, if $\bar d_i = c \bar d_j$, then $payout_f(s_i) = c \cdot payout_f(s_j)$ similarly represents the fact that a miner is paid proportionally with respect his hashrate.
		\item similarly to difficulty score, if the slice $S$ contains shares from $k_1$-th to $k_2$-th, we have \[\sum_{i=k_1}^{k_2} score_f(s_i) = 1. \] \label{addstoone}
	\end{enumerate}

\end{remark}
\subsection{Payout for each share}
When accounting for the $i$-th share, the share will belong to a slice $S$, to which a portion of block fees subsidy $F_S$ is reserved. The payout is
\[payout(s_i) = r\cdot score_d(s_i) + F_S \cdot score_f(s_i).\]
The distribution of the block subsidy necessarily depends only on $d_i$ and is independent from Job Declaration, because it is guaranteed even in empty blocks.
To verify the payouts of all shares are accurate, $N$ is the size of the window and slices are $S_1, \dots, S_t$, with $0=k_0<k_1< k_2< \dots < k_t=N$ such that the $m$-th slice contains shares $s_{k_{m-1}+1}, \dots, s_{k_{m}}$. Note, also, that $F_S = F/t$. So, by point \ref{addstoone} of Remark5 \ref{properties-diff-based-score}:
% idk how to reword this piece %
\begin{align*}
	\sum_{i=1} ^N payout(s_i) & = r\sum_{i=1}^N score_d(s_i) +F_{S}\left(\sum_{j=k_0+1}^{k_1} score_f(s_j)+\dots +\sum_{j=k_{t-1}+1}^{k_t} score_f(s_j)\right) \\
	                          & = r +F_S\cdot t                                                                                                                \\ &= r+F \\ &= R
\end{align*}
We can see that all the payouts add up to the total block reward.

\section{Shares' accountability}
A pool that uses this accounting schema is expected, for each block found, to publish all the slices
that belong to that block window calculated using PPLNS. The published slice will contain:
\begin{itemize}
	\item The number of shares contained in the slice
	\item The sum of the difficulty of the shares
	\item Fees of the slice's reference job
	\item Merkle root of the tree composed by all the shares that belong to the slice
	\item Id of the reference job
\end{itemize}
A miner must be sure that all the shares the pool reveived are valid. If the pool is also a miner, there are some incentives to pay more to itself than other miners, debasing other miners share value. One method for doing so is to produce fake shares and pay them as if they were valid. To avoid this, a miner can \emph{challenge} the pool:
\begin{enumerate}
	\item randomly select slice(s) to check; for each slice randomly select share(s) from within the slice
	\item fetch the job and the transactions that are not in the miner's cache for each selected share
	\item verify that each share is valid
	\item verify the Merkle tree for the slice with provided merkle path
	\item verify the sum of the difficulty of verified shares is not greater then the slice difficulty
	% I don't follow this one and hope to clarify %
	\item verify the fees in the shares are less than fees of the slice reference job fees + pool delta
\end{enumerate}

Whenever a miner sends a share to the pool, the pool answers with the reference job so the miner can verify if the reference job fees + delta fees are higher then the share's job fees.

If this system is implemented as an Sv2 extension, it will be easier for a miner to prove that a pool is cheating, since a miner can publish the Sv2 channel session handshake messages. This allows for non-repudiation from the pool since messages in the channel are cryptographically linked to the session.

\section{Ackowledgments}
We would like to express our gratitude to Demand \cite{demand} for their collaboration throughout the development of this project. Additionally, we extend our thanks to the developers of the open-source project \emph{Stratum v2 Reference Implementation} (\cite{sv2}), whose contributions were crucial to the success of the Sv2 project.


\appendix
\section{Motivation for slices}
In this appendix we show that if a slice is large enough, then there may be some incentives for a miner to perform a variation of pool hopping (see \cite{rosenfeld} for a definition of pool hopping). For example, we have two pools:
\begin{enumerate}
	\item POOL-1 implements a classic PPLNS
	\item POOL-2 adopts PPLNS with Job Declaration, but with the slices that coincide with the rounds of blocks mining.
\end{enumerate}
In POOL-2, if $\ell$ is a prevhash, all the shares with $\ell$ as prevhash consists of a single slice.

We calculate the payout per share of a group of miners, assuming first that they mine with POOL-1, and subsequently as if they were mining with POOL-2. Then we compare the results of these two calculations. Without loss of generality, we make some assumptions:
\begin{itemize}
	\item bitcoin difficulty is constant and all the miners have the same hashrate.
	\item the miners produce $100$ share per minute and once a block is found, the payout goes back $8$ blocks. Then we calculate the windows size for the two pools: $N = 100 shares/min \cdot 8 \cdot 10min = 8000$.
	\item the pool fee is zero and the block reward coming from the fees of mined transactions is $F$.

\end{itemize}
Since distribution of block subsidy is fixed with both methods, we assume it to be zero and we consider only the distribution of block fees $F$.
Let $payout_f(s_i)$ be the payout of the share $s_i$.

\textbf{Question:} Suppose that we want to pay the $i$-th share $s_i$. How much is paid in PPLNS-JD with full mining round sized slices compared with PPLNS?
With standard PPLNS, we have that
\[payout(s_i) = \frac{F}{8000}.\]
With PPLNS-JD with large slices, on the other hand, we have
\[payout_f(s_i) = F_s\cdot score_f(s_i),\]
where $F_S$ is part of the fee reward reserved for the slice to which the share belongs (and that contains all the shares to a specific prevhash). Since we have $8$ block, we have that $F_S = F/8$. So,
\[score_f(s_i) = \frac{\bar d_i f_i}{\sum_{j=1}^{100} \bar d_j f_j}= \frac{ f_i}{\sum_{j=1}^{100} f_j}. \]
Recall that all the miners have the same hashrate.

\textbf{Note.} From now on we assume that the growth of fees within a slice (which coincides within a mining round) is linear, namely there are $m>0$ and $c>c$ such that
\[F(f) = mt+c.\]
Futhermore, there are $100$ shares received by the pool in the time of the block to which share belong to $s_i$ (that coincides with the time frame of the slice $S$). It is safe to assume that the distribution of these shares in this timeframe is uniform. For a share $s_j$ in $S$, we have that
\[ f_j = m(6secs \cdot j) +c,\]
so
\begin{align*}
	score_f(is_i) & = \frac{m(6secs\cdot i) +c}{\sum_{j=1}^{100} m(6secs\cdot j) +c}        \\
	              & = \frac{m(6\cdot i) +c}{m\cdot 6\cdot\frac{100\cdot101}{2} +c\cdot 100} \\
	              & = \frac{m(6\cdot i) +c}{100(303\cdot m +c)}
\end{align*}
and
\[payout_f(s_i) = \frac{F}{8} \cdot \frac{m(6\cdot i) +c}{100(303\cdot m +c)}.\]
Hence, there must be a $\bar i$ such that $payout_f(s_{\bar i -1})< F/8000<payout_f(s_{\bar i })$.
Therefore, for a miner it is more profitable to mine with POOL-1 until $\bar i \cdot 6 secs$ and then join POOL-2. This is a slight variation of classic pool hopping, in which the miners jump into a pool that has just found a block and mine there for a while (this is quantified in \cite{rosenfeld}). In our case, miners will jump into this pool at the end of the mining round. This form of pool hopping is not desireable, which led to the introduction of slices. Within a slice, MMEF can be approximated as flat, so it is fair to compare a shares' fees.
It is worth pointing out that in PPLNS-JD a slice consists of all the shares with the same prevhash (produced within a single round of mining), so a miner that remains at the beginning of the round may be disadvantaged. Indeed, at the beginning his shares worth less because of the low fees (that grow linearly with time). Then, after $\bar i \cdot 6 secs$, when supposedly the fees are higher and his shares may worth more, many other hopping miners join the pool, rising the pool hashrate. This will make the shares of the faithful miner to compete against many more shares, and therefore mitigating the relief of higher fees. In conclusion, there is a double disincentive for miner to be faithful to the POOL-2.
% need to confirm my understand of this pool hopping example %

\begin{thebibliography}{9}
	\bibitem{demand}
	Demand pool \url{https://www.dmnd.work/}

	\bibitem{extension}
	Extension on share accounting
	\url{https://github.com/demand-open-source/share-accounting-ext/blob/master/extension.md}.

	\bibitem{ocean}
	Ocean pool \url{https://ocean.xyz/docs/tides}

	\bibitem{rosenfeld}
	\emph{Analysis of Bitcoin Pooled Mining Reward Systems}, M. Rosenfeld, \url{https://arxiv.org/abs/1112.4980}

	\bibitem{sv2}
	Stratum v2. Specifications: \url{https://github.com/stratum-mining/sv2-spec/} and Reference Implementation \url5{https://github.com/stratum-mining/stratum/}
\end{thebibliography}

\end{document}
